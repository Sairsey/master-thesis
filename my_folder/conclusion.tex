\chapter*{Заключение} \label{ch-conclusion}
\addcontentsline{toc}{chapter}{Заключение}	% в оглавление 
	В рамках проведённого исследования были выполнены все поставленные задачи. Были изучены различные алгоритмы симуляции мягких тел и в частности симуляции тканей. Помимо упомянутого в названии работы эврестического алгоритма, были также разработаны некоторые оптимизации для использования алгоритма XPBD на GPU. Все разработанные алгоритмы были реализованы в рамках модуля для демонстрационного проекта, использующего современные алгоритмы компьютерной графики, и поддерживающий динамическую смену алгоритма симуляции.
	
	С использованием демонстрационного проекта были проведены эксперименты, в ходе которых были показаны преимущества использования предлагаемых оптимизаций и алгоритмов перед работой библиотки Nvidia Cloth, признанной стандартом для симуляции поведения тканей. При этом, при реализации алгоритмов симуляции, был сохранен интерфейс библиотеки Nvidia Cloth. Визуальные результаты работы демонстрационного проекта были признаны качественными. 
	
	Таким образом, благодаря использованию описанных оптимизаций и сохранении интерфейса библиотеки Nvidia Cloth, предложенные алгоритмы можут быть использованы в современных графических системах, для ускорения симуляции тканей. 
	
	В качестве дальнейшей работы наиболее интересными представляются следующие два направления:
	
	Первое - поддержка тканей задаваемых готовой геометрией. Несмотря на то, что поверхности тканей можно задавать прямоугольной сеткой, наиболее простым (с точки зрения графических художников) способом задания тканей является именно при помощи готовой геометрии. В связи с этим стоит изучить дополнительные алгоритмы для определения оптимальных \say{веревок} и \say{заплаток} по не-заданной прямоугольной сеткой геометрии.
	
	Второе - адаптация предлагаемых алгоритмов для симуляции более широкого спектра мягких тел. Благодаря предлагаемым алгоритмам удалось увеличить скорость работы симуляции тканей, однако ткани являются не единственным типом мягких объектов. Предлагаемые алгоритмы позволяют оптимизировать обсчет для мягких тел задаваемых ограничениями сохранения расстояния, однако существую ограничения сохранения объема и площади, позволяющие симулировать другие мягкие тела, такие как подушки или желе.
	
	Несмотря на всё вышеперечисленное, предлагаемые алгоритмы позволяют симулировать и выводить высокодетализированные поверхности тканей, с использованием архитектуры GPU.