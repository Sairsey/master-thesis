\chapter*{Заключение} \label{ch-conclusion}
\addcontentsline{toc}{chapter}{Заключение}	% в оглавление 
	В рамках проведённого исследования были выполнены все поставленные задачи. Были изучены различные алгоритмы симуляции мягких тел и в частности симуляции тканей. Помимо упомянутого в названии работы эвристического алгоритма, были также разработаны некоторые оптимизации для применения алгоритма XPBD в реализации для графического процесора. Все разработанные алгоритмы были реализованы в рамках модуля для демонстрационного проекта, использующего современные алгоритмы компьютерной графики. Данный модуль также позволяет осуществлять динамическое переключение алгоритма симуляции.
	
	С использованием демонстрационного проекта были проведены эксперименты, в ходе которых были показаны преимущества использования предлагаемых оптимизаций и алгоритмов перед решением из библиотки Nvidia Cloth, признанной стандартом для симуляции поведения тканей. При этом, при реализации алгоритмов симуляции был сохранен интерфейс библиотеки Nvidia Cloth. Визуальные результаты работы демонстрационного проекта были признаны качественными. 
	
	Таким образом, благодаря использованию описанных оптимизаций и сохранении интерфейса библиотеки Nvidia Cloth, представленная реализация может быть использована в любых современных графических системах для ускорения симуляции тканей и улучшения её визуального качества. 
	
	В качестве дальнейшей работы наиболее интересными представляются следующие два направления:
	
	Первое - симуляция тканей, задаваемых готовыми трёхмерными объектами. Несмотря на то, что поверхности тканей можно задавать прямоугольной сеткой, наиболее простым (с точки зрения графических художников) способом их задания является именно использование готовых объектов. В связи с этим, стоит разработать дополнительные алгоритмы для определения оптимальных \say{веревок} и \say{заплаток} для произвольных полигональных сеток.
	
	Второе - адаптация предлагаемых алгоритмов для симуляции более общего представления мягких тел. Благодаря разработанным алгоритмам удалось увеличить скорость работы симуляции тканей, однако ткани являются не единственным типом мягких объектов. Предлагаемые методы позволяют оптимизировать вычисления для мягких тел, задаваемых ограничениями сохранения расстояния, однако существуют ограничения сохранения объема и площади, позволяющие симулировать другие мягкие тела, такие как органы или желе.
	
	На основании всего вышеперечисленного, предлагаемые алгоритмы позволяют симулировать и визуализировать высокодетализированные поверхности тканей, при этом обеспечивая эффективное использование вычислительных ресурсов GPU.