%%%% Начало оформления заголовка - оставить без изменений !!! %%%%
\input{my_folder/task_settings}	% настройки - начало 
	
				{%\normalfont %2020
						\MakeUppercase{\SPbPU}}\\
				\institute

\par}\intervalS% завершает input

				\noindent
				\begin{minipage}{\linewidth}
				\vspace{\mfloatsep} % интервал 	
				\begin{tabularx}{\linewidth}{Xl}
					&УТВЕРЖДАЮ      \\
					&\HeadTitle     \\			
					&\underline{\hspace*{0.1\textheight}} \Head     \\
					&<<\underline{\hspace*{0.05\textheight}}>> \underline{\hspace*{0.1\textheight}} \approveYear г.  \\  
				\end{tabularx}
				\vspace{\mfloatsep} % интервал 	
				\end{minipage}

\intervalS{\centering\bfseries%

				ЗАДАНИЕ\\
				на выполнение %с 2020 года 
				%по выполнению % до 2020 года
				выпускной квалификационной работы


\intervalS\normalfont%

				студенту \uline{\AuthorFullDat{} гр.~\group}


\par}\intervalS%
%%%%
%%%% Конец оформления заголовка  %%%%
 	
	
%add all literature
\hphantom{
	\cite{me_bachelor}
	\cite{pbd}
	\cite{xpbd}
	}
	
\begin{enumerate}[1.]
	\item Тема работы: {\expandafter \ulined \thesisTitle.}
	%\item Тема работы (на английском языке): \uline{\thesisTitleEn.} % вероятно после 2021 года
	\item Срок сдачи студентом законченной работы: \uline{\thesisDeadline.} 
	\item Исходные данные по работе: %
	\begin{itemize}
		\item Cистема для вывода трѐхмерных сцен, реализованная при помощи набора программных модулей и библиотек DirectX12 на языке С++ \cite{me_bachelor}.
		\item Инструментальные средства:
		\begin{itemize}
			\item языки программирования С++, Python
			\item среда разработки Visual Studio 2022
			\item программная библиотека для работы с видеокартой DirectX 12
			\item система контроля версий git
		\end{itemize}
	\end{itemize}
	\printbibliographyTask % печать списка источников % КОММЕНТИРУЕМ ЕСЛИ НЕ ИСПОЛЬЗУЕТСЯ
	% В СЛУЧАЕ, ЕСЛИ НЕ ИСПОЛЬЗУЕТСЯ МОЖНО ТАКЖЕ ЗАЙТИ В setup.tex и закомментировать \vspace{-0.28\curtextsize}
	\item Содержание работы (перечень подлежащих разработке вопросов):
	\begin{enumerate}[label=\theenumi\arabic*.]
		\item Введение. Обоснование актуальности
		\item Постановка задачи
		\item Обзор существующих решений 
		\item Предлагаемое решение
		\item Результаты и сравнительный анализ
		\item Заключение 
	\end{enumerate}
	\item Дата выдачи задания: \uline{\thesisStartDate.}
\end{enumerate}

\intervalS%можно удалить пробел

Руководитель ВКР \uline{\hspace*{0.1\textheight} \Supervisor}


\intervalS%можно удалить пробел

Консультант  \uline{\hspace*{0.1\textheight}\ConsultantExtra}


\intervalS%можно удалить пробел

Консультант  \uline{\hspace*{0.1\textheight}\ConsultantExtraTwo}


\intervalS%можно удалить пробел

%Консультант по нормоконтролю \uline{\hspace*{0.1\textheight} \ConsultantNorm}%ПОКА НЕ ТРЕБУЕТСЯ, Т.К. ОН У ВСЕХ ПО УМОЛЧАНИЮ

Задание принял к исполнению \uline{\thesisStartDate}

\intervalS%можно удалить пробел

Студент \uline{\hspace*{0.1\textheight}  \Author}



\input{my_folder/task_settings_restore}	% настройки - конец