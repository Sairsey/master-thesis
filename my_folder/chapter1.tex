\chapter{ПОСТАНОВКА ЗАДАЧИ} \label{ch1}

\section{Техническое задание} \label{ch1:sec1}
Требуется разработать и реализовать алгоритмы и их модификации, применимые в современных грфических конвейерах, решающие в реальном времени следующие задачи:
\begin{itemize}
	\item симуляция тканей методом XPBD;
	\item улучшение сходимости метода XPBD с помощью эвристического алгоритма;
	\item улучшение производительности метода XPBD с учетом архитектурных особенностей GPU и используемого графического конвейера;
	\item задача коллизии тканей с плоскостями, сферами, капсулами;
	\item задача коллизии тканей с другими тканями.
\end{itemize}

\section{Ожидаемый результат} \label{ch1:sec2}
Ожидаемым результатом работы является улучшение проекта \say{DX12Engine} за счет добавления в него функциональности, осуществляющей симуляцию поведения тканей. При этом, в результирующем проекте должна присутствовать возможность переключения в реальном времени алгоритма симуляции между тремя вариантами: 
\begin{itemize}
	\item XBPD использующий GPU;
	\item XPBD использующий GPU, совмещенный с эвристическим алгоритмом;
	\item алгоритм симуляции представленный в библиотеке NVidia Cloth.
\end{itemize}

Помимо этого, ожидаемым результатом является сравнительный анализ реализованных алгоритмов как с точки зрения производительности, так и с точки зрения визуальной корректности.


%\FloatBarrier % заставить рисунки и другие подвижные (float) элементы остановиться

%\section{Выводы} \label{ch1:conclusion}

%Текст выводов по главе \thechapter.

%Кроме названия параграфа <<выводы>> можно использовать (единообразно по всем главам) следующие подходы к именованию последних разделов с результатами по главам:
%\begin{itemize}
%	\item <<выводы по главе N>>, где N --- номер соответствующей главы;
%	\item <<резюме>>;
%	\item <<резюме по главе N>>, где N --- номер соответствующей главы.
%\end{itemize}

%Параграф с изложением выводов по главе \textit{является обязательным}.

%% Вспомогательные команды - Additional commands
%
%\newpage % принудительное начало с новой страницы, использовать только в конце раздела
%\clearpage % осуществляется пакетом <<placeins>> в пределах секций
%\newpage\leavevmode\thispagestyle{empty}\newpage % 100 % начало новой страницы