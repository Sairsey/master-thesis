%% Не менять - Do not modify
%%\input{my_folder/summary_settings} 
\chapter*[Count-me]{Реферат} % * - не нумеруем
\thispagestyle{empty}% удаляем параметры страницы
%\setcounter{sumPageFirst}{\value{page}}
%sumPageFirst \arabic{sumPageFirst}
%
%
%% Возможность проверить другие значения счетчиков - debugging
%\ref*{TotPages}~с.,
%\formbytotal{mytotalfigures}{рисун}{ок}{ка}{ков},
%\formbytotal{mytotaltables}{таблиц}{у}{ы}{},
%There are \TotalValue{mytotalfigures} figures in this document
%There are \TotalValue{mytotalfiguresInApp} figuresINAPP in this document
%There are \TotalValue{mytotaltables} tables in this document
%There are \TotalValue{mytotaltablesInApp} figuresINAPP in this document
%There are \TotalValue{myappendices} appendix chapters in this document
%\total{citenum}~библ. наименований.



%% Для того, чтобы значения счетчиков корректно отобразились, необходимо скомпилировать файл 2-3 раза
На \total{mypages}~c.,  
\formbytotal{myfigures}{рисун}{ок}{ка}{ков},
\formbytotal{mytables}{таблиц}{у}{ы}{},
\formbytotal{myappendices}{приложен}{ие}{ия}{ий}%.  

%\noindent
{\MakeUppercase{Ключевые слова: \keywordsRu}.}%\footnote{Всего \textbf{слов}: от 3 до 15. Всего \textbf{слов и словосочетаний}: от 3 до 5. Оформляются в именительном падеже множественного числа (или в единственном числе, если нет другой формы), оформленных по правилам русского языка. \textit{Внимание! Размещение сноски после точки является примером как запрещено оформлять сноски.}} % Ключевые слова из renames.tex

Тема выпускной квалификационной работы: <<\thesisTitle>>%\footnote{Реферат \textbf{должен содержать}: предмет, тему, цель ВКР; метод или методологию проведения ВКР; результаты ВКР; область применения результатов ВКР; выводы.}.

%%
%% Реферат ОТ 1000 ДО 1500 знаков на русский или английский текст
%%
%Реферат должен содержать:
%- предмет, тему, цель ВКР;
%- метод или методологию проведения ВКР:
%- результаты ВКР:
%- область применения результатов ВКР;
%- выводы.


Данная работа посвящена разработке и реализации эвристического алгоритма для улучшения сходимости метода Extended Position Based Dynaimics (XPBD) в задаче симуляции ткани. Данный эвристический алгоритм призван стабилизировать результат, получаемый в ходе работы метода XPBD в случае резких и сильных изменений положения частиц ткани.

Для демонстрации разработанного алгоритма были созданы демонстрационные сцены, состоящие из различных трехмерных геометрических объектов, а также симулируемых тканей. Весь исходный код проекта написан на языке C++ с применением графической библиотеки DirectX 12. Средством для программирования шейдеров является язык HLSL.

Наиболее значимым результатом является эвристический алгоритм, позволяющий избавиться от чрезмерных колебаний ткани, вызванных недостаточным количеством итераций алгоритма XPBD. В результате данных колебаний, поведение симулируемой ткани выглядит отлично от наблюдаемого в реальной жизни, и больше напоминает поведение резины. В ходе работы был реализован алгоритм XPBD, выполняющийся на GPU, а также разработаны различные алгоритмы и методы, позволяющие ускорить работу алгоритма XPBD с учетом особенностей архитектуры GPU.

Предложенный алгоритм может быть использован совместно с существующими системами симуляции тканей для более корректной визуализации в современных графических приложениях.


%\abstractRu\footnote{ОТ 1000 ДО 1500 печатных знаков (ГОСТ Р 7.0.99-2018 СИБИД) на русский или английский текст. Текст реферата повторён дважды на русском и английском языке для демонстрации подхода к нумерации страниц.} % Аннотация из renames.tex
\newpage

\printTheAbstract % не удалять

\total{mypages}~pages, 
\total{myfigures}~figures, 
\total{mytables}~tables,
\total{myappendices}~appendices%.

%\noindent
{\MakeUppercase{Keywords: \keywordsEn}.} % Ключевые слова из renames.tex 
	
The subject of the master thesis is <<\thesisTitleEn>>.
	
This paper focuses on the development and implementation of a heuristic algorithm to improve the convergence of the Extended Position Based Dynamics (XPBD) method in cloth simulation. The algorithm is specifically designed to stabilize the results of the XPBD method in cases of sudden and significant changes in the position of cloth particles. 

To demonstrate the effectiveness of the algorithm, various three-dimensional geometric objects and simulated cloth were used in demonstration scenes. The entire project was written in C++ using the DirectX 12 graphics library, with HLSL as the shader programming language. 

The most significant result of this work is the heuristic algorithm, which effectively reduces excessive cloth vibrations caused by insufficient iterations of the XPBD algorithm. These vibrations can result in unrealistic behavior of the simulated cloth, resembling that of rubber rather than real cloth. Additionally, the XPBD algorithm was implemented on the GPU and various techniques were developed to optimize its performance, taking into account the GPU architecture. 

This proposed algorithm can be integrated with existing cloth simulation systems to improve the accuracy of cloth visualization in modern graphics applications.
%\abstractEn % Аннотация из renames.tex
	


%% Не менять - Do not modify
\thispagestyle{empty}
%\setcounter{sumPageLast}{\value{page}} %сохранили номер последней страницы Задания
%\setcounter{sumPages}{\value{sumPageLast}-\value{sumPageFirst}}
%sumPageLast \arabic{sumPageLast}
%
%sumPages \arabic{sumPages}
%\restoregeometry % восстанавливаем настройки страницы
%\input{my_folder/summary_settings_restore}	% настройки - конец